\documentclass[11pt]{article}
\usepackage{amsmath,amssymb}
\usepackage{graphicx}
\usepackage{hyperref}
\usepackage{authblk}
\usepackage{listings}
\usepackage{geometry}
\geometry{margin=1in}

\title{From Fermion Leakage to Decoherence:\\ A Lattice Interpretation of Warped Extra Dimensions}
\author[1]{Justin Bilyeu}
\author[2]{Sage Research Collective}
\affil[1]{\texttt{justin@example.com}}
\affil[2]{\texttt{sage@example.com}}
\date{\today}

\begin{document}
\maketitle

\begin{abstract}
We reinterpret the leakage of brane--localized fermions into a warped extra
dimension as coherent amplitude diffusion into a discrete lattice of
possibility nodes, $\Phi_n(x)$.  By replacing the geometric coordinate $y$
with a site index $n$, we recast the 5--D Dirac equation as a tight--binding
model.  This relabeling maintains unitarity while eliminating the need for
physical extra geometry.  Through a three--site quiver model, we show how the
lowest--energy fermion mode transitions from brane--localized to delocalized
as the bulk mass $M$ exceeds a threshold.  We derive an effective decoherence
rate $\Gamma(M,k)$ for a brane--local probe, showing that leakage into the
lattice directly induces observable dissipation.  These results suggest that
extra--dimensional phenomenology can be mimicked by non--geometric information
redistribution, offering testable predictions in lattice--based analog
systems.
\end{abstract}

\section{Framework Summary}
\begin{itemize}
    \item \textbf{Motivation:} In Randall--Sundrum (RS) models, bulk fermions
    ``leak'' from a four--dimensional brane into a warped five--dimensional
    spacetime.
    \item \textbf{Proposal:} Replace the fifth dimension $y$ with a discrete
    index $n$ over a coherence lattice $\Phi_n(x)$.  Spatial leakage becomes
    amplitude diffusion.
    \item \textbf{Current conservation:} The 5--D current component $J^{y}$
    maps to a lattice derivative $\partial_n \Pi_n(x)$ that tracks possibility
    volume across nodes.
\end{itemize}

\section{Three--Site Quiver Model}
\subsection{Hamiltonian}
\[
H \;=\;
\begin{pmatrix}
M & k & 0 \\\\
k & 0 & k \\\\
0 & k & 0
\end{pmatrix},
\]
where site $n=0$ represents the brane and $n=1,2$ are coherence nodes.

\subsection{Leakage Transition}
Figure~\ref{fig:leakage} shows the off--brane probability of the ground state
as a function of bulk mass $M$ at fixed hopping $k=1$.

\begin{figure}[h]
    \centering
    \includegraphics[width=0.6\linewidth]{leakage_vs_M.png}
    \caption{Off--brane probability (sites 1+2) for the lowest--energy mode.
    The transition from localization to delocalization occurs near
    $M\!\sim\!k$.}
    \label{fig:leakage}
\end{figure}

\section{Leakage Phase Diagram}
A sweep over $M\in[0,4]$ and $k\in[0,2]$ yields the heat map in
Fig.~\ref{fig:phase}, revealing the bound, transitional, and delocalized
regions.

\begin{figure}[h]
    \centering
    \includegraphics[width=0.7\linewidth]{leakage_phase.png}
    \caption{Leakage probability across parameter space.  Dark wedge: bound
    zero--mode.  Gradient band: leakage onset.  Bright plateau: full
    delocalization.}
    \label{fig:phase}
\end{figure}

\section{Decoherence Rate $\Gamma(M,k)$}
Coupling a brane--local probe to the first two coherence nodes produces an
effective Lindblad rate
\[
\Gamma(M,k)\;=\;\sum_{n=1}^{2} \frac{g_n^{2}}{\Lambda_n^{2}}\,|\psi_n|^{2},
\]
with representative couplings $g_1=1$, $g_2=0.8$, $\Lambda_{1,2}=1$.
The resulting heat map is shown in Fig.~\ref{fig:gamma}.

\begin{figure}[h]
    \centering
    \includegraphics[width=0.7\linewidth]{gamma_phase.png}
    \caption{Decoherence rate $\Gamma(M,k)$ (arbitrary units) over the same
    parameter grid.}
    \label{fig:gamma}
\end{figure}

\section{Significance \& Outlook}
\begin{itemize}
    \item \textbf{Geometry--free extra dimensions:} Extra--dimensional leakage
    can be reproduced by non--geometric information redistribution.
    \item \textbf{Experimental handle:} $\Gamma(M,k)$ provides a falsifiable
    link to cold--atom or superconducting--qubit platforms.
    \item \textbf{Next steps:} Extend to larger lattices to recover KK--like
    towers, animate amplitude drift, and fit decoherence rates to specific
    analogue systems.
\end{itemize}

\section*{Code Listing}
\begin{lstlisting}[language=Python, basicstyle=\ttfamily\small]
# Python snippet: compute Gamma(M,k) on a 3-site lattice
import numpy as np
import matplotlib.pyplot as plt

def gamma(M, k, g1=1.0, g2=0.8, Lambda1=1.0, Lambda2=1.0):
    H = np.array([[M,k,0],[k,0,k],[0,k,0]], float)
    evals, evecs = np.linalg.eigh(H)
    psi = evecs[:, np.argmin(np.abs(evals))]
    p1, p2 = abs(psi[1])**2, abs(psi[2])**2
    return g1**2/Lambda1**2 * p1 + g2**2/Lambda2**2 * p2
\end{lstlisting}

\section*{Acknowledgements}
We thank the ResonanceGeometry community for insightful discussions.

\bibliographystyle{unsrt}
\bibliography{refs}
\end{document}
